\documentclass[11pt, leqno]{article}
\usepackage{amsmath,amssymb,amsthm}
\usepackage{algorithm}
\usepackage[noend]{algpseudocode} 
\usepackage{cancel}

%---enable russian----

\usepackage[utf8]{inputenc}
\usepackage[russian]{babel}

% PROBABILITY SYMBOLS
\newcommand*\PROB\Pr 
\DeclareMathOperator*{\EXPECT}{\mathbb{E}}


% Sets, Rngs, ets 
\newcommand{\N}{{{\mathbb N}}}
\newcommand{\Z}{{{\mathbb Z}}}
\newcommand{\R}{{{\mathbb R}}}
\newcommand{\Zp}{\ints_p} % Integers modulo p
\newcommand{\Zq}{\ints_q} % Integers modulo q
\newcommand{\Zn}{\ints_N} % Integers modulo N

% Landau 
\newcommand{\bigO}{\mathcal{O}}
\newcommand*{\OLandau}{\bigO}
\newcommand*{\WLandau}{\Omega}
\newcommand*{\xOLandau}{\widetilde{\OLandau}}
\newcommand*{\xWLandau}{\widetilde{\WLandau}}
\newcommand*{\TLandau}{\Theta}
\newcommand*{\xTLandau}{\widetilde{\TLandau}}
\newcommand{\smallo}{o} %technically, an omicron
\newcommand{\softO}{\widetilde{\bigO}}
\newcommand{\wLandau}{\omega}
\newcommand{\negl}{\mathrm{negl}} 

% Misc
\newcommand{\eps}{\varepsilon}
\newcommand{\inprod}[1]{\left\langle #1 \right\rangle}

\newcommand{\handout}[5]{
  \noindent
  \begin{center}
  \framebox{
    \vbox{
      \hbox to 5.78in { {\bf Научно-исследовательская практика} \hfill #2 }
      \vspace{4mm}
      \hbox to 5.78in { {\Large \hfill #5  \hfill} }
      \vspace{2mm}
      \hbox to 5.78in { {\em #3 \hfill #4} }
    }
  }
  \end{center}
  \vspace*{4mm}
}

\newcommand{\lecture}[4]{\handout{#1}{#2}{#3}{Scribe: #4}{Название темы #1}}

\newtheorem{theorem}{Теорема}
\newtheorem{lemma}{Лемма}
\newtheorem{definition}{Определение}
\newtheorem{corollary}{Следствие}
\newtheorem{fact}{Факт}

% 1-inch margins
\topmargin 0pt
\advance \topmargin by -\headheight
\advance \topmargin by -\headsep
\textheight 8.9in
\oddsidemargin 0pt
\evensidemargin \oddsidemargin
\marginparwidth 0.5in
\textwidth 6.5in

\parindent 0in
\parskip 1.5ex


\begin{document}

\lecture{"Обобщение Эйлера теоремы Ферма"}{Лето 2020}{}{Чубань Артем}

\qquad Доказательство может быть проиллюстрировано с некоторыми конкретными числами. Пусть $n = 9$, для примера. Положительные простые числа, взаимно простые с 9, тогда
$$1, 2, 4, 5, 7, 8.$$
Они играют роль целых $a_1, a_2, ..., a_{\phi(n)}$ в доказательстве Теоремы 7-5. Если $a = -4$, то тогда целые $aa_i$ будут
$$-4, -8, -16, -20, -28, -32,$$
где, по модулю 9,
$$-4 \equiv 5, -8 \equiv 1, -16 \equiv 2, -20 \equiv 7, -28 \equiv 8, -32 \equiv 4.$$
Когда мы умножим все вышеперечисленные сравнения вместе, получим
$$(-4)(-8)(-16)(-20)(-28)(-32) \equiv 5 \cdot 1 \cdot 2 \cdot 7 \cdot 8 \cdot 4 (mod\ 9),$$
что становится
$$(1 \cdot 2 \cdot 4 \cdot 5 \cdot 7 \cdot 8)(-4)^6 \equiv (1 \cdot 2 \cdot 4 \cdot 5 \cdot 7 \cdot 8)(mod\ 9)$$
Будучи взаимнопростыми с 9, шесть целых чисел 1, 2, 4, 5, 7, 8 могут быть успешно упразднены, чтобы получить
$$(-4)^6 \equiv 1 (mod\ 9)$$
Достоверность этого последнего сравнения подтверждается следующим вычислением
$$(-4)^6 \equiv 4^6 \equiv (64)^2 \equiv 1^2 \equiv 1 (mod\ 9)$$

\qquad Заметим, что теорема 7-5 действительно обобщает теорему Ферма, которую мы доказали ранее. Ведь если p - простое число, то тогда $\phi (p) = p - 1$; следовательно, когда НОД$(a, p) = 1$, мы получаем
$$a^{p - 1} \equiv a^{\phi(p)} \equiv 1 (mod\ p)$$
и так:

\corollary
Если p - просто число и $p\ \cancel{|}\ a$, то тогда $a^{p - 1} \equiv 1(mod\ p)$.

\subparagraph{Пример 7-2}
Теорема Эйлера полезна для уменьшения больших степеней по модулю n. Приведем простой пример, найдем последние две цифры в десятичном представлении числа $3^{256}$; что эквивалетно нахождению наименьшего неотрицательного целого числа, которое сравнимо с $3^{256}$ по модулю $100$. Так как НОД$(3, 100) = 1$ и
$$\phi(100) = \phi(2^5 \cdot 5^2) = 100\left(1 - \frac{1}{2}\right) \left( 1 - \frac{1}{5} \right) = 40,$$
по теореме Эйлера получаем
$$3^{40} \equiv 1 (mod\ 100).$$
По алгоритму деления, $256 = 6 \cdot 40 + 16$; откуда
$$3^{256} \equiv 3^{6 \cdot 40 + 16} \equiv \left( 3^{40} \right)^6 3^{16} \equiv 3^{16}(mod\ 100)$$
и наша задача свелась к одной оценке $3^{16}$, по модулю $100$. Вычисления следующие, с опущенными размышлениями:
$$3^{16} \equiv (81)^4 \equiv (-19)^4 \equiv (361)^2 \equiv 61^2 \equiv 21 (mod\ 100)$$

Существует и другой путь к теореме Эйлера, который требует использование теоремы Ферма.

\subparagraph{Второе доказательство теоремы Эйлера}
Для начала, докажем методом индукции, что если $p \cancel{|} a$ (p и a - простые), то тогда
\begin{equation}
a^{\phi \left(p^k \right)} \equiv 1 (mod\ p^k), \qquad k>0.
\end{equation}
При $k = 1$, данное выражение сводится к утверждению теоремы Ферма. Предположим, что (1) верно при некотором k, докажем, что оно верно и при $k + 1$.\\
Поскольку предпологается, что (1) верно, то мы можем записать
$$a^{\phi(p^k)} = 1 + qp^k$$
для некоторого целого q. Заметим также, что
$$\phi(p^{k + 1}) = p^{k + 1} - p^k = p(p^k - p^{k-1}) = p \phi(p^k)$$
Используя эти факты ывместе с биномиальной теоремой, получаем

\begin{flalign*}
	a^{\phi(p^{k + 1})} & =  a^{p \phi(p^k)} &
	\cr & =  (1 + qp^k)^p &
	\cr & =  1 + \binom{p}{1}(qp^k) + \binom{p}{2}(qp^k)^2 + ... + \binom{p}{p - 1}(qp^k)^{p - 1} + (qp^k)^p &
	\cr & =  1 + \binom{p}{1}(qp^k)(mod\ p^{k + 1}) &
\end{flalign*}
Однако $p\ |\ \binom{p}{1}$, поэтому $p^{k + 1}\ |\ \binom{p}{1}(qp^k)$. Таким образом, последнее написанное сравнение превращается в
$$a^{\phi(p^{k+1})} \equiv 1 (mod\ p^{k + 1}),$$
завершая этап введения.

\qquad Теперь пусть НОД$(a, n) = 1$ и n уже имеет простую факторизацию $n = p_1^{k_1} p_2^{k_2} ... p_r^{k_r}$. Ввиду доказанного, каждое из сравнений
\begin{equation}
a^\phi(p_i^{k_i}) \equiv 1 (mod\ p_i^{k_i}), \qquad i = 1, 2, ..., r.
\end{equation}
Взаимная простота модулей приводит нас к соотношению
$$a^{\phi(n)} \equiv 1 (mod\ p_1^{k_1} p_2^{k_2} ... p_r^{k_r})$$
или $a^{\phi(n)} \equiv 1 (mod\ n)$

\qquad Полезность теоремы Эйлера в теории чисел трудно преувеличить. Она приводит, например, другое доказательство китайской теоремы об остатке. Другими словами, мы стремимся обосновать, что если НОД$(n_i, n_j) = 1$ для $i \neq j$, то тогда система линейных сравнений
$$x \equiv a_i (mod\ n_i), \qquad i = 1, 2, ..., r$$
допускает совместное решение. Пусть $n = n_1 n_2 ... n_r$ и $N_i = n\ |\ n_i$ для $i = 1, 2, ..., r$. Тогда целое число
$$x = a_1 N_1^{\phi(n_1)} + a_2 N_2^{\phi(n_2)} + ... + a_r N_r^{\phi(n_r)}$$
выполняет наши требования. Чтобы увидеть это, для начала заметим, что $N_j \equiv 0 (mod\ n_i)$ всякий раз, когда $i \neq j$; откуда
$$x \equiv a_i N_i^{\phi(n_i)} (mod\ n_i)$$
Однако, когда НОД$(N_i, n_i) = 1$, мы имеем
$$N_i^{\phi(n_i)} \equiv 1 (mod\ n_i)$$
и поэтому $x \equiv a_i (mod\ n_i)$ для любого i.

\qquad В качестве второго применения теоремы Эйлера покажем, что если n-нечетное целое число, которое не кратно 5, то n делит целое число, все цифры которого равны 1. (Например: $7\ |\ 111111$.) Так как НОД$(n, 10) = 1$ и НОД$(9, 10 = 1)$, мы имеем НОД$(9n, 10) = 1$. Снова цитируя теорему 7-5,
$$10^{\phi(9n)} \equiv 1 (mod\ 9n).$$
Это говорит о том, что $10^{\phi(9n)} - 1 = 9nk$ для некоторого целого числа k, что эквивалентно
$$kn = \frac{10^{\phi(9n)} - 1}{9}$$
Правая часть приведенного выражения представляет собой целое число, все цифры которого равны 1, причем каждая цифра числителя явно равна 9.

\begin{center}
	\item \paragraph{PROBMLEMS 7.3}
\end{center}

\begin{enumerate}
	\item Используя теорему Эйлера, доказать следующее:
		\begin{enumerate}
			\item Для любых целых a, $a^{37} \equiv a (mod\ 1729). [$Подсказка: $1729 = 7 \cdot 13 \cdot 19.]$
			\item Для любых целых a, $a^{13} \equiv a (mod\ 2730). [$Подсказка: $2730 = 2 \cdot 3 \cdot 5 \cdot 7 \cdot 13.]$
			\item Для любых нечетных целых a, $a^{33} \equiv a (mod\ 4080). [$Подсказка: $4080 = 15 \cdot 16 \cdot 17.]$
		\end{enumerate}
	\item Показать, что если НОД$(a, n) =$ НОД$(a-1, n) = 1$, то тогда
	$$1 + a + a^2 + ... + a^{\phi(n)-1}  \equiv 0 (mod\ n).$$
	$[$Подсказка: Вспомните, что $a^{\phi(n)} - 1 \equiv (a - 1)(a^{\phi(n) - 1} + .. + a^2 + a + 1). ]$
	\item Если m и n - взаимнопростые положительные целые числа, доказать
	$$m^{\phi(n)} + n^{\phi(m)} \equiv 1 (mod\ mn)$$
	\item Заполните пропущенные детали в данном доказатльстве теоремы Эйлера: Пусть p - простой делитель для n и НОД$(a, p) = 1$. По теореме Ферма, $a^{p - 1} \equiv 1 (mod\ p)$, поэтому $a^{p - 1} = 1 + tp$, для некоторого t. Тогда $a^{p(p - 1)} = (1 + tp)^p = 1 + \binom{p}{1} (tp) + ... + (tp)^p \equiv 1 (mod\ p^2)$ и, по индукции, $a^{p^{k - 1}(p - 1)} \equiv 1 (mod\ p^k)$, где $k = 1, 2, ...\ .$ Возведем обе стороны этого сравненияв степень $\phi(n)\ |\ p^{k - 1}(p - 1)$, чтобы получить $a^{\phi(n)} \equiv 1 (mod\ p^k)$. Таким образом, $a^{\phi(n)} \equiv 1 (mod\ n)$.
	\item Найти цифру разряда единиц числа $3^{100}$ с помощью теоремы Эйлера.
	\item 	\begin{enumerate}
			\item Если НОД$(a, n) = 1$, показать что линейное сравнение $ax \equiv b (mod\ n)$ имеет решение $x \equiv  ba^{\phi(n) - 1} (mod\ n)$.
			\item Используя часть (a), решить сравнения $3x \equiv 5 (mod\ 26)$, $13x \equiv 2 (mod\ 40)$ и $10x \equiv 21 (mod\ 49)$.
			\end{enumerate}
	\item докаnm, что каждое простое число, кроме $2$ и $5$, делит бесконечно много целых чисел $1, 11, 111, ...$ .
\end{enumerate}

\end{document}