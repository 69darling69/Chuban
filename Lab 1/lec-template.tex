\documentclass[11pt, leqno]{article}
\usepackage{amsmath,amssymb,amsthm}
\usepackage{algorithm}
\usepackage[noend]{algpseudocode} 
\usepackage{cancel}

%---enable russian----

\usepackage[utf8]{inputenc}
\usepackage[russian]{babel}

% PROBABILITY SYMBOLS
\newcommand*\PROB\Pr 
\DeclareMathOperator*{\EXPECT}{\mathbb{E}}


% Sets, Rngs, ets 
\newcommand{\N}{{{\mathbb N}}}
\newcommand{\Z}{{{\mathbb Z}}}
\newcommand{\R}{{{\mathbb R}}}
\newcommand{\Zp}{\ints_p} % Integers modulo p
\newcommand{\Zq}{\ints_q} % Integers modulo q
\newcommand{\Zn}{\ints_N} % Integers modulo N

% Landau 
\newcommand{\bigO}{\mathcal{O}}
\newcommand*{\OLandau}{\bigO}
\newcommand*{\WLandau}{\Omega}
\newcommand*{\xOLandau}{\widetilde{\OLandau}}
\newcommand*{\xWLandau}{\widetilde{\WLandau}}
\newcommand*{\TLandau}{\Theta}
\newcommand*{\xTLandau}{\widetilde{\TLandau}}
\newcommand{\smallo}{o} %technically, an omicron
\newcommand{\softO}{\widetilde{\bigO}}
\newcommand{\wLandau}{\omega}
\newcommand{\negl}{\mathrm{negl}} 

% Misc
\newcommand{\eps}{\varepsilon}
\newcommand{\inprod}[1]{\left\langle #1 \right\rangle}

\newcommand{\handout}[5]{
  \noindent
  \begin{center}
  \framebox{
    \vbox{
      \hbox to 5.78in { {\bf Научно-исследовательская практика} \hfill #2 }
      \vspace{4mm}
      \hbox to 5.78in { {\Large \hfill #5  \hfill} }
      \vspace{2mm}
      \hbox to 5.78in { {\em #3 \hfill #4} }
    }
  }
  \end{center}
  \vspace*{4mm}
}

\newcommand{\lecture}[4]{\handout{#1}{#2}{#3}{Scribe: #4}{Название темы #1}}

\newtheorem{theorem}{Теорема}
\newtheorem{lemma}{Лемма}
\newtheorem{definition}{Определение}
\newtheorem{corollary}{Следствие}
\newtheorem{fact}{Факт}

% 1-inch margins
\topmargin 0pt
\advance \topmargin by -\headheight
\advance \topmargin by -\headsep
\textheight 8.9in
\oddsidemargin 0pt
\evensidemargin \oddsidemargin
\marginparwidth 0.5in
\textwidth 6.5in

\parindent 0in
\parskip 1.5ex


\begin{document}

\lecture{"Обобщение Эйлера теоремы Ферма"}{Лето 2020}{}{Чубань Артем}

\qquad Доказательство может быть проиллюстрировано с некоторыми конкретными числами. Пусть $n = 9$, для примера. Положительные простые числа, взаимно простые с 9, тогда
$$1, 2, 4, 5, 7, 8.$$
Они играют роль целых $a_1, a_2, ..., a_{\phi(n)}$ в доказательстве Теоремы 7-5. Если $a = -4$, то тогда целые $aa_i$ будут
$$-4, -8, -16, -20, -28, -32,$$
где, по модулю 9,
$$-4 \equiv 5, -8 \equiv 1, -16 \equiv 2, -20 \equiv 7, -28 \equiv 8, -32 \equiv 4.$$
Когда мы умножим все вышеперечисленные сравнения вместе, получим
$$(-4)(-8)(-16)(-20)(-28)(-32) \equiv 5 \cdot 1 \cdot 2 \cdot 7 \cdot 8 \cdot 4 (mod\ 9),$$
что становится
$$(1 \cdot 2 \cdot 4 \cdot 5 \cdot 7 \cdot 8)(-4)^6 \equiv (1 \cdot 2 \cdot 4 \cdot 5 \cdot 7 \cdot 8)(mod\ 9)$$
Будучи взаимнопростыми с 9, шесть целых чисел 1, 2, 4, 5, 7, 8 могут быть успешно упразднены, чтобы получить
$$(-4)^6 \equiv 1 (mod\ 9)$$
Достоверность этого последнего сравнения подтверждается следующим вычислением
$$(-4)^6 \equiv 4^6 \equiv (64)^2 \equiv 1^2 \equiv 1 (mod\ 9)$$

\qquad Заметим, что теорема 7-5 действительно обобщает теорему Ферма, которую мы доказали ранее. Ведь если p - простое число, то тогда $\phi (p) = p - 1$; следовательно, когда НОД$(a, p) = 1$, мы получаем
$$a^{p - 1} \equiv a^{\phi(p)} \equiv 1 (mod\ p)$$
и так:

\corollary
Если p - просто число и $p\ \cancel{|}\ a$, то тогда $a^{p - 1} \equiv 1(mod\ p)$.

\subparagraph{Пример 7-2}
Теорема Эйлера полезна для уменьшения больших степеней по модулю n. Приведем простой пример, найдем последние две цифры в десятичном представлении числа $3^{256}$; что эквивалетно нахождению наименьшего неотрицательного целого числа, которое сравнимо с $3^{256}$ по модулю $100$. Так как НОД$(3, 100) = 1$ и
$$\phi(100) = \phi(2^5 \cdot 5^2) = 100\left(1 - \frac{1}{2}\right) \left( 1 - \frac{1}{5} \right) = 40,$$
по теореме Эйлера получаем
$$3^{40} \equiv 1 (mod\ 100).$$
По алгоритму деления, $256 = 6 \cdot 40 + 16$; откуда
$$3^{256} \equiv 3^{6 \cdot 40 + 16} \equiv \left( 3^{40} \right)^6 3^{16} \equiv 3^{16}(mod\ 100)$$
и наша задача свелась к одной оценке $3^{16}$, по модулю $100$. Вычисления следующие, с опущенными размышлениями:
$$3^{16} \equiv (81)^4 \equiv (-19)^4 \equiv (361)^2 \equiv 61^2 \equiv 21 (mod\ 100)$$

Существует и другой путь к теореме Эйлера, который требует использование теоремы Ферма.

\subparagraph{Второе доказательство теоремы Эйлера}
Для начала, докажем методом индукции, что если $p \cancel{|} a$ (p и a - простые), то тогда
\begin{equation}
a^{\phi \left(p^k \right)} \equiv 1 (mod\ p^k), \qquad k>0.
\end{equation}
При $k = 1$, данное выражение сводится к утверждению теоремы Ферма. Предположим, что (1) верно при некотором k, докажем, что оно верно и при $k + 1$.\\
Поскольку предпологается, что (1) верно, то мы можем записать
$$a^{}$$




\end{document}